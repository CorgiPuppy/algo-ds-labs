\documentclass[12pt, a4paper]{report}
\usepackage[top=1cm, left=1cm, right=1cm]{geometry}

\usepackage[utf8]{inputenc}
\usepackage[russian]{babel}

\usepackage{array}
\newcolumntype{M}[1]{>{\centering\arraybackslash}m{#1}}

\usepackage{hyperref}
\hypersetup{
	colorlinks,
	citecolor=black,
	filecolor=black,
	linkcolor=black,
	urlcolor=black
}

\usepackage{sectsty}
\allsectionsfont{\centering}

\usepackage{indentfirst}
\setlength\parindent{24pt}

\usepackage{algorithm}
\usepackage[noend]{algpseudocode}

\usepackage{listings}
\usepackage{xcolor}
\definecolor{codegreen}{rgb}{0,0.6,0}
\definecolor{codegray}{rgb}{0.5,0.5,0.5}
\definecolor{codepurple}{rgb}{0.58,0,0.82}
\definecolor{backcolour}{rgb}{0.95,0.95,0.92}
\lstdefinestyle{mystyle}{
    backgroundcolor=\color{backcolour},
    commentstyle=\color{codegreen},
    keywordstyle=\color{magenta},
    numberstyle=\normalsize\color{codegray},
    stringstyle=\color{codepurple},
    basicstyle=\ttfamily\footnotesize,
    breakatwhitespace=false,
    breaklines=true,
    captionpos=b,
    keepspaces=true,
    numbers=left,
    numbersep=5pt,
    showspaces=false,
    showstringspaces=false,
    showtabs=false,
    tabsize=2
}

\usepackage{graphicx}
\graphicspath{ {plots/pictures/}{assets/pictures} }

\begin{document}
	\begin{titlepage}
		\begin{center}
			\large \textbf{Министерство науки и высшего образования Российской Федерации} \\
			\large \textbf{Федеральное государственное бюджетное образовательное учреждение высшего образования} \\
			\large \textbf{«Российский химико-технологический университет имени Д.И. Менделеева»} \\

			\vspace*{4cm}
			\LARGE \textbf{ОТЧЕТ ПО ЛАБОРАТОРНОЙ РАБОТЕ №4}

			\vspace*{4cm}
			\begin{flushright}
				\Large
				\begin{tabular}{>{\raggedleft\arraybackslash}p{9cm} p{10cm}}
					Выполнил студент группы КС-36: & Золотухин А.А. \\
					Ссылка на репозиторий: & https://github.com/ \\
					& MUCTR-IKT-CPP/ \\
					& ZolotukhinAA\_36\_ALG \\
					Принял: & Крашенников Роман Сергеевич \\
					Дата сдачи: & 24.03.2025 \\
				\end{tabular}
			\end{flushright}

			\vspace*{6cm}
			\Large \textbf{Москва \\ 2025}
		\end{center}
	\end{titlepage}

	\tableofcontents
	\thispagestyle{empty}
	\newpage

	\pagenumbering{arabic}

	\section*{Описание задачи}
	\addcontentsline{toc}{section}{Описание задачи}
	\large
	В рамках лабораторной работы необходимо реализовать генератор случайных графов, генератор должен содержать следующие параметры:
	\begin{itemize}
		\item максимальное/минимальное количество генерируемых вершин;
		\item максимальное/минимальное количество генерируемых рёбер;
		\item максимальное количество рёбер, связанных с одной вершиной;
		\item генерируется ли направленный граф;
		\item максимальное количество входящих и выходящих рёбер.
	\end{itemize}
	\par
	Сгенерированный граф должен быть в рамках одного класса (этот класс не должен заниматься генерацией) и должен обладать обязательно следующими методами:
	\begin{itemize}
		\item выдача матрицы смежности;
		\item выдача матрицы инцидентности;
		\item выдача списка смежности;
		\item выдача списка рёбер.
	\end{itemize}
	\par
	В качестве проверки работоспособности требуется сгенерировать 10 графов с возрастающим количеством вершин и рёбер (количество выбирать в зависимости от сложности расчёта для вашего отдельно взятого ПК)
	\par
	На каждом из сгенерированных графов требуется выполнить поиск кратчайшего пути или подтвердить его отсутствие из точки А в точку Б, выбирающиеся случайным образом заранее, поиском в ширину и поиском в глубину, замерев время, требуемое на выполнение операции. Результаты замеров наложить на график и проанализировать эффективность применения обоих методов к этой задаче.
	
	\newpage

	\section*{Описание метода/модели}
	\addcontentsline{toc}{section}{Описание метода/модели}
	\large
	\textbf{Вершина графа} - некоторая точка, связанная с другими точками. \par
	\textbf{Ребро графа} - линия, соединяющая две точки и олицетворяющая связь между ними. \par
	\textbf{Граф} - множество вершин, соединённых друг с другом произвольным образом множеством рёбер. \par
	\textbf{Ориентированным графом} называют такой граф, в котором каждое ребро имеет направление движения, и, как правило, не преполагает возможности обратного перемещения. \par
	\textit{Описание графа}. Для описания графа используют один из следующих вариантов:
	\begin{itemize}
		\item \textbf{Матрица смежности} - двумерная таблица, для которой столбцы соответствуют вершинам, а значения в таблице соответствуют рёбрам, для \textit{невзвешенного} графа они могут быть просто 1, если связь есть и идёт в нужном направлении, и 0, если её нет, а для \textit{невзвешенного} графа будут стоять конкретные значения.
		\item \textbf{Матрица инцидентности} - матрица, в которой строки соответствуют вершинам, а столбцы соответствуют связям, и в ячейке ставится 1, если связь выходит из вершины, -1, если входит, и 0 во всех остальных случаях.
		\item \textbf{Список смежности} - список списков, содержащий все вершины, а внутренние списки для каждой вершины содержат все смежные ей.
		\item \textbf{Список рёбер} - список строк, в которых хранятся все рёбра вершины, а внутреннее значение содержит две вершины, к котором присоединено это ребро.
	\end{itemize}
	\par
	\textbf{Обход графа} - переход от одной его вершины к другой в поисках свойств связей этих вершин. Выделяют два варианта обхода: обход в глубину и обход в ширину. \par
	\textbf{DFS (deep first search)} следует концепции "погружайся глубже, в головой вперёд". Идея заключается в том, что мы двигаемся от начальной вершины в определённом направлении до тех пор, пока не достигнем конца пути. Если мы достигли конца пути, но он не является пунктом назначения, то мы возвращаемся назад (к точке разветвления) и идём по другому маршруту. \par
	\textbf{BFS (breadth first search)} следует концепции "расширяйся, поднимаясь на высоту птичьего полёта". Вместо того, чтобы двигаться по определённом пути до конца, BFS предполагает движение вперёд по одному соседу за раз, вместо следования по пути, BFS подразумевает посещение ближайших к вершине соседей за одно действие, затем посещение соседей соседей и так до тех пор, пока не будет обнаружена исковая вершина или соседи не закончатся.

	\newpage

	\section*{Выполнение задачи}
	\addcontentsline{toc}{section}{Выполнение задачи}

	\newpage
	\vfill

	\begin{figure}
		\includegraphics[width=300pt]{bfs_dfs.png}
	\end{figure}

	\begin{figure}[h]
		\centering
		\includegraphics[width=1\textwidth]{graph_1.png}
		\caption{Граф 1.}
	\end{figure}
	\begin{figure}[h]
		\centering
		\includegraphics[width=1\textwidth]{graph_2.png}
		\caption{Граф 2.}
	\end{figure}
	\begin{figure}[h]
		\centering
		\includegraphics[width=1\textwidth]{graph_3.png}
		\caption{Граф 3.}
	\end{figure}
	\begin{figure}[h]
		\centering
		\includegraphics[width=1\textwidth]{graph_4.png}
		\caption{Граф 4.}
	\end{figure}
	\begin{figure}[h]
		\centering
		\includegraphics[width=1\textwidth]{graph_5.png}
		\caption{Граф 5.}
	\end{figure}

	\vfill
	\clearpage

	\section*{Выводы}
	\addcontentsline{toc}{section}{Выводы}

\end{document}
