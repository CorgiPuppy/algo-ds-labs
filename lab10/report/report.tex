\documentclass[12pt, a4paper]{report}
\usepackage[top=1cm, left=1cm, right=1cm]{geometry}

\usepackage[utf8]{inputenc}
\usepackage[russian]{babel}

\usepackage{array}
\newcolumntype{M}[1]{>{\centering\arraybackslash}m{#1}}

\usepackage{hyperref}
\hypersetup{
	colorlinks,
	citecolor=black,
	filecolor=black,
	linkcolor=black,
	urlcolor=black
}

\usepackage{sectsty}
\allsectionsfont{\centering}

\usepackage{indentfirst}
\setlength\parindent{24pt}

\usepackage{algorithm}
\usepackage[noend]{algpseudocode}

\usepackage{listings}
\usepackage{xcolor}
\definecolor{codegreen}{rgb}{0,0.6,0}
\definecolor{codegray}{rgb}{0.5,0.5,0.5}
\definecolor{codepurple}{rgb}{0.58,0,0.82}
\definecolor{backcolour}{rgb}{0.95,0.95,0.92}
\lstdefinestyle{mystyle}{
    backgroundcolor=\color{backcolour},
    commentstyle=\color{codegreen},
    keywordstyle=\color{magenta},
    numberstyle=\normalsize\color{codegray},
    stringstyle=\color{codepurple},
    basicstyle=\ttfamily\footnotesize,
    breakatwhitespace=false,
    breaklines=true,
    captionpos=b,
    keepspaces=true,
    numbers=left,
    numbersep=5pt,
    showspaces=false,
    showstringspaces=false,
    showtabs=false,
    tabsize=2
}

\usepackage{graphicx}
\graphicspath{ {plots/pictures/}{assets/pictures} }

\begin{document}
	\begin{titlepage}
		\begin{center}
			\large \textbf{Министерство науки и высшего образования Российской Федерации} \\
			\large \textbf{Федеральное государственное бюджетное образовательное учреждение высшего образования} \\
			\large \textbf{«Российский химико-технологический университет имени Д.И. Менделеева»} \\

			\vspace*{4cm}
			\LARGE \textbf{ОТЧЕТ ПО ЛАБОРАТОРНОЙ РАБОТЕ №10}

			\vspace*{4cm}
			\begin{flushright}
				\Large
				\begin{tabular}{>{\raggedleft\arraybackslash}p{9cm} p{10cm}}
					Выполнил студент группы КС-36: & Золотухин Андрей Александрович \\
					Ссылка на репозиторий: & https://github.com/ \\
					& MUCTR-IKT-CPP/ \\
					& ZolotukhinAA\_36\_ALG \\
					Принял: & Крашенников Роман Сергеевич \\
					Дата сдачи: & 12.05.2025 \\
				\end{tabular}
			\end{flushright}

			\vspace*{6cm}
			\Large \textbf{Москва \\ 2025}
		\end{center}
	\end{titlepage}

	\tableofcontents
	\thispagestyle{empty}
	\newpage

	\pagenumbering{arabic}

	\section*{Описание задачи}
	\addcontentsline{toc}{section}{Описание задачи}
	\large
	
	\newpage

	\section*{Описание метода/модели}
	\addcontentsline{toc}{section}{Описание метода/модели}
	\large
	Идея \textbf{динамического программирования} в том, что мы разбиваем задачу на малые подзадачи, спускаясь вниз, далее, находим самую минимальную подзадачу, решаем её, затем, имея решение, мы создаём подзадачи, которые включают в своё решение решённую более меньшую подзадачу, снова получае их решения, сохраняем их, и движемся далее вверх. \par
	По итогу динамическое программирование требует следующий условий для решений какой-либо задачи:
	\begin{itemize}
		\item перекрывающиеся подзадачи;
		\item оптимальная подструктура;
		\item возможность запоминания решения часто встречающихся подзадач.
	\end{itemize}

	\section*{Выполнение задачи}
	\addcontentsline{toc}{section}{Выполнение задачи}
	Задача реализована на языке \textit{Java}.
	\lstset{style=mystyle}
	\lstinputlisting[language=Java]{src/main/java/com/dp/Main.java}
\end{document}
